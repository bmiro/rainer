\section{Joc de proves}

Per tal de provar la pràctica s'han realitzat una sèrie de jocs de proves
descrits a continuació.

El primer es simple, passar per quatre punts sense cap obstacle. Amb aquest simplement es comprova que es calculi
bé el vector d'atracció, també es pot observar si el heading es fa de manera adequada.

En segon lloc tenim passar per quatra punts pero amb un obstacle enmig, en aquest es veu si l'esquiva d'obstacles
es fa bé. A més l'obstacle esta colocat de tal manera que el robot es troba impactant una linia de 45 graus
amb el punt destí a la perpendicular, de tal manera que seria una situació delicada on quedar-se estancat.

A continuació trobam el mateix obstacle però aquest cop situat sobre el punt de destí, en aquest cas
el robot ha de detectar que no pot arribar-hi ja que l'obstacle i el punt de destí es tan per davall d'un cert
llindar. Fins que no s'assoleix aquest llindar es veu com el robot intenta fer aproximacions al punt desde
diferents direccions.

Per tal de provar el vagar tenim un mapa tancat amb diversos obstacles on el robot va rebotant.
Por arribar un punt en que el robot entri en un cicle ja que els angles de rebot poden fer que així
coincideixi. El que no es permet es que el robot toqui un obstacle o es quedi imòbil sols girant sobre
ell mateix sense ser capaç de emprendre la ruta cap a una altra banda.

A continuació tenim les proves de la neteja de zona, la primera sense obstacles per veure el recorregut.

I finalment el recorregut amb obstacles on el robot els esquiva (i marca terreny durant aquesta esquiva)
i com finalment torna a les zones d'obstacle per comprovar que no vos un obstàcle mòbil i ara si que
pot netejar la zona. En aquest últim no podem simular obstacles mòbils però si veure com torna a intentar
anar a la zona obstaculitzada.

