\section{Modelització i implementació}

Tal i com s'explica a l'enunciat podem distignir tres nivells en l'arquitectura: estrategic, tàctic i executiu.

El nivell executiu esta implementat en les llibreries de l'Aria, així doncs la nostra pràctica implementa
els dos nivells superiors.

El nivell que més atenció requereix es el tàctic ja que l'estratègic actua com a mer sequenciador
de funcions implementades en el tàctic.

En el codi la llibreria de l'estratègic es anomenada librainer i la tàctica libtact.

De manera transversal hi ha altres components com el mapa que el nivell tàctic omple amb les zones
per on es va passat, així com a l'estratègic per obtenir els seguents punts on ordenar que vagi el robot.
Dit component apareix al codi com una llibreria anomenada librainermap.

També existeix una altre llibreria emprada transversalment que és lib2d, en aquesta s'han implementat
dues estructures de dades i les seves operacions per tal de facilitar tota la feina als demes componets.
En aquesta llibreria hi trobam totes les operacions necessàries per operar amb punts i vectors bidimensionals
com ara les operacions aritmètiques clàssiques, productes per un escalar, normalitzat etc.

Finalment hi ha una llibreria no emprada al codi anomenada libtrace que pretenia identificar llocs
innaccessibles envoltats per objectes pero finalment no s'ha acabat de implementar ja que s'ha considerat
fora de l'abast d'aquesta pràctica.

DIAGRAMA CAPES LIBRAINER LIBTACT ETC


En la implementació el més destacable es com es du a terme l'esquiva dels obstacles i com s'ha jugat amb
els paràmetres per evitar que el robot es bloqueji o tardi massa temps en fer un recorregut. 

En l'elaboració del vector de repulsió es tenen amb compte tots els sensors. En principi es podria pensar
que basten els de davant ja que son els que detecten l'objecte, però es interessant posar els de darrera
ja que en el moment que intenta fer l'esquiva els de davant perden la lectura mentre que els de darrera
la guanyen i així s'afavoreix un distanciament més rapid i perpendicular de l'objecte. La ponderació
de cada un dels vectors s'ha fet de manera experimental amb el simulador, es pot dir que els frontals
tenen mes rellevància que els de darrera, i que les diagonals tenen també major ponderació ja que 
s'ha observat que aixi s'esquiven millor els cantons prims com per exemple d'un triangle.

El problema de l'esquiva d'un obstacle pot ser que els vectors quedin igualats i el robot bloquejat
o fent uns petits moviments ja que no te temps avançar perque a l'haver-se reoirentat no detecta l'obstacles
i intenta  tornar al punt d'on venia perque es mes proper a l'objectiu. Per això s'implementa un temps de
ceguera (blindTime) que el robot no mira els sensors i es mou en la direcció previament calculada.
D'aquesta manera quan el robot s'apopa a un obstacle i la repulsió
es suficient aquest es gira en sentit oposat i avança.

Per altra banda es permet que el robot avanci mentre gira cosa que no sols agilitza el moviment sino que
modfica la trajectoria del robot evitant que torni al mateix punt disminuint les possiblitats de bloqueig.

Cal esmentar que això no garanteix que es pugui esquivar qualsevol obstacle per arribar a un punt, 
si l'obstacle es suficientment gran es possible que el robot vagi de banda a banda d'aquest de manera
cíclica sense aconseguir superar-lo. En l'apartat d'ampliacions es comenta una possible so\lgem ució.

En l'apartat seguent s'expliquen els paràmetres que governen aquest comportament, que son el temps
de ceguesa i el llindar que considera que el robot ja esta orientat per poder moure's.

La implementació del vagar es la típica de avançar fins a trobar un obstacle i un cop es te posar 
la nova direcció del robot com el vector de repulsió generat per aquest obstacle.

Finalment comentar que per tal de simplificar l'entorn es modelitza la zona a netejar com una retícula
de ce\lgem d'igual dimensió caracteritzades per les coordenades del seu centre i l'estat en que es troben,
si netes, brutes o son una zona ocupada per un obstacle.


COMENTAR IMPLEMENTACIÓ DE LES TASQUES

En la implementació trobam que el robot comença la neteja d'una zona, la ce\lgem a a netejar es simplement
el resultat de fer un zig-zag per tota la zona, independentment si esta neta o no. Alhora hi ha dues tasques
iniciades que van registrat per on ha passat el robot i per tant consideren que s'ha netejat la zona i una altre
que observa els valors dels sensors per la detecció d'obstacles. Cal dir que aquesta segona sols mira els sensors,
no fa res més, aquest fet està comentat a l'apartat de possibles ampliacions.

Els obstacles es marquen a nivell estratègic quan el nivell tàctic indica que no ha pogut arribar a aquell punt perque conicideix 
amb un objecte. El fet de no fer-ho dins la tasca de mapeig es perque un obstacle pot ser suficientment petit
com per representar un problema a l'aproximació desde un sol costat, així doncs si es fa a nivell de la funció
anar a punt del tàctic aquesta l'intenta vorejar amb l'evitació d'obstacles, si es marques amb la simple deteció
ja es consideraria una cela amb obstacle i s'aniria a la següent.
 