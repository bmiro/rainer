\section{Modelització i implementació}

\subsection{Arquitectura}
Tal i com s'explica a l'enunciat podem distingir tres nivells en l'arquitectura: estratègic, tàctic i executiu.

El nivell executiu està implementat en les llibreries de l'\emph{Aria}, així doncs la nostra pràctica implementa
els dos nivells superiors.

El nivell que més atenció requereix és el tàctic ja que l'estratègic actua com a mer seqüenciador
de funcions implementades en el tàctic.

En el codi la llibreria de l'estratègic es anomenada \texttt{librainer} i la tàctica \texttt{libtact}.

De manera transversal hi ha altres components com el mapa que el nivell tàctic omple amb les zones
per on es va passat, així com a l'estratègic per obtenir els següents punts on ordenar que vagi el robot.
Dit component apareix al codi com una llibreria anomenada \texttt{librainermap}.

També existeix una altra llibreria emprada transversalment que és \texttt{lib2d}, en aquesta s'han implementat
dues estructures de dades i les seves operacions per tal de facilitar tota la feina als demés components.
En aquesta llibreria hi trobam totes les operacions necessàries per operar amb punts i vectors bidimensionals
com ara les operacions aritmètiques clàssiques, productes per un escalar, normalitzat etc.

Finalment hi ha una llibreria no emprada al codi anomenada \texttt{libtrace} que pretenia identificar llocs
inaccessibles envoltats per objectes però finalment no s'ha acabat de implementar ja que s'ha considerat
fora de l'abast d'aquesta pràctica.


\begin{center}
  \begin{tabular}{|c|c|c|}
    \hline
    \multirow{3}{*}{lib2d} & \multicolumn{2}{|c|}{rainer}  \\
    \cline{2-3}\cline{3-3}
			  & librainer & \multirow{2}{*}{librainermap}\\
    \cline{2-2}
			  & libtact & \\
    \hline
  \end{tabular}
\end{center}


Així doncs en el nivell tàctic tenim les funcions per anar a un punt (i funcions auxiliars per el càlcul
de vectors) i la funció de vagar. Al mateix nivell s'executen dues tasques en para\lgem el una per
enregistrar els punts dels objectes trobats i una altre per enregistrar en tot moment la posició del 
robot i les zones per on ha passat marcant-les com a netes.

\subsection{Implementació de la repulsió}

En la implementació el més destacable és com es du a terme l'esquiva dels obstacles i com s'ha jugat amb
els paràmetres per evitar que el robot és bloquegi o tardi massa temps en fer un recorregut. 

El càlcul de cada un dels components és la implementació mencionada a l'enunciat, cada sensor que
detecta l'obstacle a menys d'un determinat llindar genera un vector del punt on s'ha detectat l'obstacle
al centre del robot. Aquests vectors són normalitzats, ponderats segons quin sigui el sensor i finalment
sumats.

Després és sumant ponderadament amb el vector d'atracció per tal d'obtenir la direcció resultant.

En l'elaboració del vector de repulsió es tenen amb compte tots els sensors. En principi es podria pensar
que basten els de davant ja que són els que detecten l'objecte, però es interessant posar els de darrera
ja que en el moment que intenta fer l'esquiva els de davant perden la lectura mentre que els de darrera
la guanyen i així s'afavoreix un distanciament més ràpid i perpendicular de l'objecte. 

La ponderació de cada un dels vectors s'ha fet de manera experimental amb el simulador, es pot dir
que els frontals tenen més rellevància que els de darrera, i que les diagonals tenen també major
ponderació ja que s'ha observat que així s'esquiven millor els cantons prims com per exemple d'un triangle.

El problema de l'esquiva d'un obstacle pot ser que els vectors quedin igualats i el robot bloquejat
o fent uns petits moviments ja que no te temps avançar perquè a l'haver-se re-orientat no detecta l'obstacle
i intenta tornar al punt d'on venia perquè és més proper a l'objectiu. Per això s'implementa un temps de
ceguera (\emph{blindTime}) que el robot no mira els sensors i es mou en la direcció prèviament calculada.
D'aquesta manera quan el robot s'apropa a un obstacle i la repulsió es suficient aquest es gira en sentit
oposat i avança.

Per altra banda es permet que el robot avanci mentre gira cosa que no sols agilitza el moviment sinó que
modifica la trajectòria del robot evitant que torni al mateix punt disminuint les possibilitats de bloqueig.

Cal esmentar que això no garanteix que es pugui esquivar qualsevol obstacle per arribar a un punt, 
si l'obstacle es suficientment gran és possible que el robot vagi de banda a banda d'aquest de manera
cíclica sense aconseguir superar-lo. En la secció d'ampliacions es comenta una possible so\lgem ució. En
la secció següent s'expliquen els paràmetres que governen aquest comportament, que són el temps
de ceguesa i el llindar que considera que el robot ja esta orientat per poder moure's.

\subsection{Implementació del vagar}

La implementació del vagar consisteix en avançar fins a trobar un obstacle i un cop es té posar 
la nova direcció del robot com el vector de repulsió generat per aquest obstacle.

\subsection{Implementació de la zona de neteja}

Finalment comentar que per tal de simplificar l'entorn és modelitza la zona a netejar com una retícula
de ce\lgem d'igual dimensió caracteritzades per les coordenades del seu centre i l'estat en que es troben,
si netes, brutes o son una zona ocupada per un obstacle.

En la implementació trobam que el robot comença la neteja d'una zona, la ce\lgem a a netejar és simplement
el resultat de fer un zig-zag per tota la zona, independentment si esta neta o no. Alhora hi ha dues tasques
iniciades que van registrat per on ha passat el robot i per tant consideren que s'ha netejat la zona, l'altra
observa els valors dels sensors per la detecció d'obstacles. Cal dir que aquesta segona sols mira els sensors,
no fa res més, aquest fet està comentat a l'apartat de possibles ampliacions.

Els obstacles es marquen a nivell estratègic quan el nivell tàctic indica que no ha pogut arribar a aquell
punt perquè coincideix amb un objecte. El fet de no fer-ho dins la tasca de mapeig és perquè un obstacle
pot ser suficientment petit com per representar un problema a l'aproximació des de un sol costat, així doncs
si es fa a nivell de la funció \emph{anar a punt} del tàctic aquesta l'intenta vorejar amb l'evitació d'obstacles, 
si en canvi es marques amb la simple tasca de detecció ja es consideraria una ce\lgem a amb obstacle i s'aniria a la següent.

Un cop s'ha fet tot el zig-zag es marquen per tal de comprovar si els obstacles no eren transitoris es marquen
com ce\lgem es brutes \ref{neteja} i és cerca la més propera al robot, aquest intenta accedir-hi i determina si l'ha netejat o es definitivament un
obstacle\ref{netejaobs}. Tot seguit cerca el següent punt més proper per dur a terme el mateix proces fins a donar per verificats tots els punts.

 