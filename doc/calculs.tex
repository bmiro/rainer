\section{Càlculs}
En aquest apartat explicam com ho fa el robot per calcular la direcció que a de
seguir i com ho fa per evitar els obstacles. Existeixen diferents tècniques,
però com indica l'enunciat em utilitzat comportaments reactius, més concretament
els camps de potencial. 

Un comportament és un mòdul que a partir d'una sèrie d'estimuls dóna una
resposta determinada. Com em comentat aquests comportaments són reactius, per
tant no es realitza cap tipus de planificació i només depen de les dades que es
tenen en cada instant. 

Existeixen diferents formes d'implementar els comportaments. A n'aquesta
pràctica em utilitzat els camps de potencial. Els camps de potencial són un
mètode que consisteix en modelitzar el robot com si fos una partícula sotmesa a
forces. Cada comportament generara una força i la direcció del robot s'obtindrà
a partir de la suma ponderada del resultat de cada comportament.

Els comportaments que s'han implementat són el d'anar cap a un objectiu i el
d'evitar obstacles. El primer produeix un vector d'atracció cap al punt
objectiu. El segon, per cada sensor que detecta un objecte produeix un vector
de repulsió. Llavors, aquets dos vectors es combinen i s'obté el vector
resultant que ens indicarà la direcció a la qual s'ha de dirigir el robot. 

\subsection{Càlcul vector d'atracció}

\subsection{Càlcul vector de repulsió}

\subsection{Càlcul vectro director} 




