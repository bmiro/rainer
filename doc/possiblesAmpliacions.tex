\section{Possibles ampliacions}

En aquest apartat es mencionen possibles ampliacions de la practica actual. El fet de fer aquestes mencions
es perque així es demostra la flexibilitat de l'arquitectura implementada que alhora mostra el grau
de comprensió adquirit.

El fet de no implementar aquestes ampliacions es perquè donada l'arquitectura de base es senzill però pot resultar
rutinari i alhora costós en temps per els pocs coneixements diferents del que ja s'ha implementat que es demostren.

Una primera millora seria implementar un viatgant de comerç per passar a fer la neteja de tots els punts.
Per fer això simplement seria necessari implementar la funció que generés la ruta òptima, guardar-la en 
memòria i implementar un iterador perque cada crida de la funció getNextCell reconrés el punt on s'ha d'anar.
Amb aquesta implementació s'evitarien recorregunts innecessaris sobre ce\lgem es ja netejades i el recorregut
mínim per fer-ho.

No s'ha implementat ja que el cost es NP complet i el fet de trobar un obstacle pot esgarrar facilment la ruta.

Per altra banda seria senzill implementar una espera d'events de teclat perque el robot atengui una urgència,
simplement podria fer-se amb una tasca nova pendent d'events de teclat que sobreesriqui el valor del  punt
on ha d'anar el robot.

Notem que com que la tasca de posicionar el robot al mapa i de mirar que s'ha netejat son una tasca apartat
a l'atendre una emergencia també es considerarien nets els punts per on ha passat per atendre-la.

Finalment també es podria treballar el fet de trobar llocs innaccessibles. Una opció per fer-ho
es un cop iniciat el moviment a un punt guardar en un vector els punts per on es passa i quan passa
cert llidar de temps mirar si la posició actual es un punt ja dins del vector, així podriem comprovar
si esta movent-se de costat a costat intentant accedir a un lloc on no es possible.

Finalment també seria interessant que la tasca que vigila el valors dels sensors escrivis a un mapa de bits
les lectures fetes, així es tindria una imatge amb el mapa de la situació. Aquesta funcionalitat 
en l'estat actual de la practica es molt senzilla de implementar, pero requereix tenir soltura emprant
mapes de bits i generant imatges.