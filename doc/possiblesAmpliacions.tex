\section{Possibles ampliacions}

En aquest apartat es mencionen possibles ampliacions de la pràctica actual. El fet de fer aquestes mencions
és perquè així es demostra la flexibilitat de l'arquitectura implementada i alhora mostra el grau
de comprensió adquirit veient com es podria continuar treballant.

El fet de no implementar aquestes ampliacions és perquè és costós en temps per
els pocs coneixements que es demostren.

\subsection{Ruta de neteja òptima}
Una primera millora seria implementar un viatjant de comerç per passar a fer la neteja de tots els punts.
Per fer això simplement seria necessari implementar la funció que generés la ruta òptima, guardar-la en 
memòria i implementar un iterador perquè cada crida de la funció \texttt{getNextCell} recorrér el punt on s'ha d'anar.
Dit iterador és una simple vector amb un punter també conegut com \emph{array vulgaris}.
Amb aquesta implementació s'evitarien recorreguts innecessaris sobre ce\lgem es
ja netejades i el recorregut mínim per fer-ho.

Hem de tenir amb compte que el cost és NP i el fet de trobar un obstacle pot esgarrar fàcilment la ruta.

\subsection{Atenció d'emergències}
Per altra banda seria senzill implementar una espera d'esdeveniments de teclat perquè el robot atengui una urgència,
simplement podria fer-se amb una tasca nova pendent d'esdeveniments de teclat que sobreescrigui el valor del  punt
on ha d'anar el robot.

Notem que com que la tasca de posicionar el robot al mapa i de mirar que s'ha netejat són una tasca apartat
a l'atendre una emergència també es considerarien nets els punts per on ha passat per atendre-la.

\subsection{Inaccessibilitat}
Finalment també es podria treballar el fet de trobar llocs inaccessibles. Una opció per fer-ho
és un cop iniciat el moviment a un punt guardar en un vector els punts per on
es passa i quan passa cert llindar de temps mirar si la posició actual es un
punt ja dins del vector, així podríem comprovar si està movent-se de costat a
costat intentant accedir a un lloc on no és possible.

Dita funcionalitat sols requereix una lleu modificació a la tasca que registra
els punts per on s'està passat i una variable de comunicació.

\subsection{Imatge del mapa}
Finalment també seria interessant que la tasca que vigila el valors dels sensors
escrivís a un mapa de bits les lectures fetes, així es tindria una imatge amb el
mapa de la situació. Aquesta funcionalitat en l'estat actual de la pràctica és
molt senzilla de implementar, però requereix tenir soltura emprant mapes de bits
i generant imatges.

A l'igual que l'anterior seria una modificació a la tasca que registra els
punts per on s'està passant.